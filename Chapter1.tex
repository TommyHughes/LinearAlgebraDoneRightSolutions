\chapter{Vector Spaces}

\section{ \label{1.A} \(\mathbb{R}^n\) and \(\mathbb{C}^n\)}

\begin{exercise} \label{1.A.1}
    Suppose \( a \) and \( b \) are real numbers, not both \( 0 \). Find real numbers \( c \) and \( d \) such that
    \[ \frac{1}{a+bi} = c+di \]
    \begin{proof}
        \begin{align*}
            \frac{1}{a+bi} &= \left( \frac{1}{a+bi} \right) \left( \frac{a-bi}{a-bi} \right) \\
            &= \frac{a-bi}{a^2+b^2} \\
            &= \frac{a}{a^2+b^2} + \frac{-b}{a^2+b^2}i \\
            &= c+di
        \end{align*}
    \end{proof}
\end{exercise}

\begin{exercise} \label{1.A.2}
    Show that
    \[ \frac{-1+\sqrt{3}i}{2} \]
    is a cube roote of \( 1 \) (meaning that its cube equals 1).
    
    \begin{proof}
        \begin{align*}
            \left[ \frac{-1+\sqrt{3}i}{2} \right]^3 &= \frac{(-1)^3+3(-1)^2(\sqrt{3}i)+3(-1)(\sqrt{3}i)^2+(\sqrt{3}i)^3}{8} \\
            &= \frac{-1 + 3\sqrt{3}i+9-3\sqrt{3}i}{8} \\
            &= \frac{8}{8} \\
            &= 1
        \end{align*}
    \end{proof}
\end{exercise}

\begin{exercise} \label{1.A.3}
    Find two distinct square roots of \( i \).
    
    \begin{proof}
        Notice that 
        \begin{align*}
            \left[ \frac{\sqrt{2}+\sqrt{2}i}{2} \right]^2 &= \frac{(\sqrt{2})^2+2(\sqrt{2})(\sqrt{2}i)+(\sqrt{2}i)^2}{4} \\
            &= \frac{2 + 4i - 2}{4} \\
            &= i
        \end{align*}
        Indeed,\( -\frac{\sqrt{2}+\sqrt{2}i}{2} \) produces the same result.
    \end{proof}
    
    \textbf{Note}: This solution was not arrived at by trial and error or intuition. Using the well-known formula
    \[ e^{i\theta} = \cos \theta + i \sin \theta \]
    and setting \( cos \theta =0 \) and \( \sin \theta = 1 \) produces \( \theta = \frac{\pi}{2} + n2\pi \). Using this value for \( \theta \) and taking the square root produces the solution.
\end{exercise}

\begin{exercise} \label{1.A.4}
    Show that \( \alpha + \beta = \beta + \alpha \) for all \( \alpha, \beta \in \mathbb{C} \).
    
    \begin{proof}
        Let \( \alpha = a_1+a_2i \) and \( \beta = b_1 + b_2i \).
        \begin{align*}
            \alpha + \beta &= \left( a_1+a_2i \right) + \left( b_1+b_2i \right) \\
            &= \left( a_1+b_1 \right) + \left( a_2+b_2 \right) i \\
            &= \left( b_1+a_1 \right) + \left( b_2+a_2 \right) i \\
            &= \left( b_1+b_2i \right) + \left( a_1+a_2i \right) \\
            &= \beta + \alpha
        \end{align*}
    \end{proof}
\end{exercise}

\begin{exercise} \label{1.A.5}
    Show that \( (\alpha + \beta) + \lambda = \alpha + (\beta + \lambda) \) for all \( \alpha, \beta, \lambda \in \mathbb{C} \).
        
    \begin{proof}
        \begin{align*}
            (\alpha + \beta) + \lambda &= ((a_1+a_2i) + (b_1+b_2i)) + (c_1+c_2i) \\
            &= ((a_1+b_1)+(a_2+b_2)i) + (c_1+c_2i) \\
            &= (a_1+b_1+c_1) + (a_2+b_2+c_2)i \\
            &= (a_1+a_2i) + ((b_1+c_1)+(b_2+c_2)i) \\
            &= (a_1 + a_2i) + ((b_1+b_2i)+(c_1+c_2i)) \\
            &= \alpha + (\beta + \lambda)
        \end{align*}
    \end{proof}
\end{exercise}

\begin{exercise} \label{1.A.6}
    Show that \( (\alpha \beta)\lambda = \alpha(\beta \lambda) \) for all \( \alpha,\beta, \lambda \in \mathbb{C} \).
    
    \begin{proof}
        Similar to above with more tedious calculation.
    \end{proof}
\end{exercise}

\begin{exercise} \label{1.A.7}
    Show that for every \( \alpha \in \mathbb{C} \), there exists a unique \( \beta \in \mathbb{C} \) such that \( \alpha + \beta = 0 \).
    
    \begin{proof}
        If \( \alpha = a_1+a_2i \) and \( \beta = b_1+b_2i \) then
        \[ \alpha + \beta = (a_1+a_2i) + (b_1+b_2i) = (a_1+b_1)+(a_2+b_2)i = 0+0i \]
        if and only if
        \[
        \begin{cases}
        a_1+b_1 = 0 \\
        a_2+b_2 = 0
        \end{cases}
        \]
        if and only if \( b_i = -a_i \). Therefore, the unique solution is \( b= -a_1-a_2i \).
    \end{proof}
\end{exercise}

\begin{exercise} \label{1.A.8}
    Show that for every \( \alpha \in \mathbb{C} \) with \( \alpha \neq 0 \), there exists a unique \( \beta \in \mathbb{C} \) such that \( \alpha \beta = 1 \).
    
    \begin{proof}
        If \( \alpha = (a+bi) \neq 0 \) and \( \beta = \frac{a-bi}{a^2+b^2} \) then
        \[
        \alpha \beta = (a+bi) \left( \frac{a-bi}{a^2+b^2} \right) = \frac{a^2+b^2}{a^2+b^2} = 1
        \]
        
        To demonstrate uniqueness, we see that if \( \alpha \beta_1 = 1 = \alpha \beta_2 \) then
        \begin{align*}
            \alpha \beta_1 &= \alpha \beta_2 \\
            (\beta_1 \alpha) \beta_1 &= (\beta_1 \alpha) \beta_2 \\
            \beta_1 &= \beta_2
        \end{align*}
    \end{proof}
\end{exercise}

\begin{exercise} \label{1.A.9}
    Show that \( \lambda(\alpha+\beta)=\lambda\alpha + \lambda\beta \) for all \( \lambda,\alpha,\beta \in \mathbb{C} \)
    
    \begin{proof}
        Trivial.
    \end{proof}
\end{exercise}

\begin{exercise} \label{1.A.10}
    Find \( x \in \mathbb{R}^n \) such that
    \[ (4,-3,1,7) + 2x = (5,9,-6,8) \]
    
    \begin{proof}
        Trivial.
    \end{proof}
\end{exercise}

\begin{exercise} \label{1.A.11}
    Explain why there does not exist \( \lambda \in \mathbb{C} \) such that
    \[ \lambda (2-3i, 5+4i, -6+7i) = (12-5i, 7+22i, -32-9i) \]
    
    \begin{proof}
        We see that 
        \[ \lambda(2-3i) = 12-5i \Leftrightarrow \lambda = \frac{39+26i}{13} \]
        and
        \[ \lambda(5+4i) = 7+22i \Leftrightarrow \lambda = 3+2i \]
        Therefore, there is no such \( \lambda \).
    \end{proof}
\end{exercise}

\begin{exercise} \label{1.A.12}
    Show that \( (x+y)+z = x+(y+z)\) for all \( x,y,z \in \mathbb{F}^n \).
    
    \begin{proof}
        Trivial.
    \end{proof}
\end{exercise}

\begin{exercise} \label{1.A.13}
    Show that \( (ab)x=a(bx) \) for all \( x \in \mathbb{F}^n \) and all \( a,b \in \mathbb{F} \)
    
    \begin{proof}
        Trivial.
    \end{proof}    
\end{exercise}

\begin{exercise} \label{1.A.14}
    Show that \( 1x=x \) for all \( x \in \mathbb{F}^n \).
    
    \begin{proof}
        Trivial.
    \end{proof}
\end{exercise}

\begin{exercise} \label{1.A.15}
    Show that \( \lambda(x+y)=\lambda x + \lambda y \) for all \( \lambda \in \mathbb{F} \) and all \( x,y \in \mathbb{F}^n \).
    
    \begin{proof}
        Trivial.
    \end{proof}
\end{exercise}

\begin{exercise} \label{1.A.16}
    Show that \( (a+b)x=ax+bx \) for all \( a,b \in \mathbb{F} \) and \( x \in \mathbb{F}^n \).
    
    \begin{proof}
        Trivial.
    \end{proof}
\end{exercise}

\section{\label{1.B} Definition of Vector Space}

\begin{exercise} \label{1.B.1}
    Prove that \( -(-v) = v \) for every \( v \in V \).
    
    \begin{proof}
        \( -v \) denotes the additive inverse of \( v \). This means that
        \[
        v+(-v)=-v+v=0 
        \]
        That is, \( v = -(-v) \).
    \end{proof}
\end{exercise}

\begin{exercise} \label{1.B.2}
    Suppose \( a \in \mathbb{F}, v \in V \) and \( av = 0 \). Prove that \( a = 0 \) or \( v = 0 \).
    
    \begin{proof}
        If \( a = 0 \) we are done. So suppose \( a \neq 0 \). Since \( \mathbb{F} \) is a field it follows \( a^{-1} \) exists. So
        \begin{align*}
            a v &= 0 \\
            a^{-1} (a v) &= a^{-1} 0 \\
            (a^{-1} a) v &= 0 \\
            v &= 0
        \end{align*}
    \end{proof}
\end{exercise}

\begin{exercise} \label{1.B.3}
    Suppose \( v,w \in V \). Explain why there exists a unique \( x \in V \) such that \( v + 3x = w \).
    
    \begin{proof}
        \begin{align*}
            v+3x &= w \\
            -v+(v+3x) &= -v+w \\
            (-v+v)+3x &= w-v \\
            3x &= w-v \\
            \left( \frac{1}{3} \right)(3x) &= \left( \frac{1}{3} \right)(w-v) \\
            \left( \frac{1}{3} \cdot 3 \right) x &= \left( \frac{1}{3} \right)(w-v) \\
            x &= \left( \frac{1}{3} \right)(w-v)
        \end{align*}
        For uniqueness, we see that
        \begin{align*}
            v+3x_1 &= v+3x_2 \\
            \intertext{adding by \( -v \) yields} \\
            3x_1 &= 3x_2 \\
            \intertext{multiplying by \( 1/3 \) yields} \\
            x_1 &= x_2
        \end{align*}
        Thus \( x = \left( \frac{1}{3} \right) (w-v) \) is the unique solution.
    \end{proof}
\end{exercise}

\begin{exercise} \label{1.B.4}
    The empty set is not a vector space. The empty set fails to satisfy only one of the vector space requirements. Which one?
    
    \begin{proof}
        The additive identity axiom would require that there exist \( 0 \in \emptyset \).
    \end{proof}
\end{exercise}

\begin{exercise} \label{1.B.5}
    Show that in the definition of a vector space the additive inverse condition can be replaced with the condition that 
    \[ 0v=0 \text{ for all } v \in V \]
    Here the \( 0 \) on the left side is the number \( 0 \), and the \( 0 \) on the right side is the additive identity of \( V \).
    
    \begin{proof}
        We already have that the additive inverse condition implies the stated condition (1.29). Now we must establish that the stated condition, along with the other axioms, implies the additive inverse condition. To that end, let \( 0v = 0 \) for all \( v \in V \). Then for every \( v \in V \) we get that
        \[ v + (-1)v = (1+(-1))v = 0v =0 \]
        so that we can denote \( (-1)v = -v \) as the additive inverse.
    \end{proof}
\end{exercise}

\begin{exercise} \label{1.B.6}
    Let \( \infty \) and \( -\infty \) denote two distinct objects, neither of which is in \( \mathbb{R} \). Define an addition and scalar multiplication on \( \mathbb{R} \cup \{\infty\} \cup \{-\infty\} \) as you could guess from the notation. Speceifically, the sum and product of two real numbers is as usual, and for \( t \in \mathbb{R} \) define
    \[
    t \infty = 
    \begin{cases}
    -\infty & \text{ if } t < 0 \\
    0 & \text{ if } t = 0 \\
    \infty & \text{ if } t > 0
    \end{cases}
    \]
    
    \[
    t(-\infty) = 
    \begin{cases}
    \infty & \text{ if } t < 0 \\
    0 & \text{ if } t = 0 \\
    -\infty & \text{ if } t > 0
    \end{cases}
    \]
    
    \[t + \infty = \infty + t = \infty, \hspace{15mm} t + (-\infty) = (-\infty) + t = -\infty \]
    \[\infty + \infty = \infty, \hspace{10mm} (-\infty) + (-\infty) = -\infty, \hspace{10mm} \infty + (-\infty) = 0 \]
    
    Is \( \mathbb{R} \cup \{\infty\}  \cup \{-\infty\} \) a vector space over \( \mathbb{R} \)? Explain.
    
    \begin{proof}
        
    \end{proof}
\end{exercise}

\section{\label{1.C} Subspaces}

\begin{exercise} \label{1.C.1}
    
\end{exercise}

\begin{exercise} \label{1.C.2}
    
\end{exercise}

\begin{exercise} \label{1.C.3}
    
\end{exercise}

\begin{exercise} \label{1.C.4}
    
\end{exercise}

\begin{exercise} \label{1.C.5}
    
\end{exercise}

\begin{exercise} \label{1.C.6}
    
\end{exercise}

\begin{exercise} \label{1.C.7}
    
\end{exercise}

\begin{exercise} \label{1.C.8}
    
\end{exercise}

\begin{exercise} \label{1.C.9}
    
\end{exercise}

\begin{exercise} \label{1.C.10}
    
\end{exercise}

\begin{exercise} \label{1.C.11}
    
\end{exercise}

\begin{exercise} \label{1.C.12}
    
\end{exercise}

\begin{exercise} \label{1.C.13}
    
\end{exercise}

\begin{exercise} \label{1.C.14}
    
\end{exercise}

\begin{exercise} \label{1.C.15}
    
\end{exercise}

\begin{exercise} \label{1.C.16}
    
\end{exercise}

\begin{exercise} \label{1.C.17}
    
\end{exercise}

\begin{exercise} \label{1.C.18}
    
\end{exercise}

\begin{exercise} \label{1.C.19}
    
\end{exercise}

\begin{exercise} \label{1.C.20}
    
\end{exercise}

\begin{exercise} \label{1.C.21}
    
\end{exercise}

\begin{exercise} \label{1.C.22}
    
\end{exercise}

\begin{exercise} \label{1.C.23}
    
\end{exercise}

\begin{exercise} \label{1.C.24}
    
\end{exercise}